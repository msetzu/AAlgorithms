\section{Hashing sets}

Your company has a database $S \subseteq U$ of keys. For this database, it uses a hash function $h$ uniformly chosen at random from a universal family $\mathcal{H}$ (as seen in class); it also keeps a bit vector $B_S$ of $m$ entries, initialized to zeroes, which are then set $B_S[h(k)] = 1$ for every $k \in S$ (note that collisions may happen). Unfortunately, the database $S$ has been lost, thus only $B_S$ and $h$ are known, and the rest is no more accessible. Now, given $k \in U$, how can you establish if $k$ was in $S$ or not? What is the probability of error? [Note: you are not choosing $k$ and $S$ randomly as the they are both given... randomization here is in the choice of $h \in \mathcal{H}$ performed when building $B_S$.] Under the hypothesis that $m \geq c|S|$ for some $c > 1$, find the expected number of 1s in $B_S$ under a uniform choice at random of $h \in \mathcal{H}$.

\vspace{0.5cm}
\paragraph{Solution.}
Trivially for $B_s[h(k)] = 0$ we can answer \textsc{false} with $\prob{error} = 0$.
Let us analyse the opposite case, $B_s[h(k)] = 1$.
Let $i \in [0,m]$ be some index s.t. $B_s[i] = 1$, and let $cl_{S}(i)$ be the list of $k \in S: h(k) = 1$ for some set $S \in \displaystyle {\mathcal {P}}(S)$.
We can then denote the sets $cl_{U} := \{k_{U} : k_{U} \in U, h(k) = i\}, cl_{S} := \{k_{S} : k_{S} \in S, h(k) = i\}$; it is trivial to show that
    \begin{itemize}
    \label{6_cl_inclusion} \item $cl_{S} \subseteq cl_{U}$ as $S \in U$.
    \label{6_cl_length} \item $| cl_{S}(k) | \leq | cl_{U}(k) | \forall k \in U$ as $S \subseteq U$.
    \end{itemize}
Let us now try and estimate $|cl_{S}(k)|$: given $h$ is universal, we have an expected number of collisions of $\E{X_{k}} \approx \frac{1}{m} \forall k \in S$, that is
\begin{align*}
    \prob{h(k^{0}) = c} = \frac{1}{m}                               					\\
    \prob{h(k^{0}) = c, h(k^{1}) = c} = \left(\frac{1}{m}\right)^{2}                    \\
    \prob{h(k^{j}) = c, \dots, h(k^{j + i} = c)} = \left(\frac{1}{m}\right)^{j - i}    	\\
    \prob{h(k) = c, \forall k \in S} = \left(\frac{1}{m}\right)^{|S|}
\end{align*}
We can similarly compute the probability of not collision by simply replacing $\frac{1}{m}$ with $(1 - \frac{1}{m})$:
\begin{equation*}
  \prob{h(k) != c, \forall k \in S} = \left(1 - \frac{1}{m}\right)^{|S|}
\end{equation*}

Given our estimate of the collision list, we can now compute an estimate of the error probability: by~\ref{6_cl_inclusion}, we give an erroneous answer whenever $k \in cl_{U}(k), k \notin cl_{S}(k)$, that is we have a margin of error of $cl_{U}(k) \setminus cl_{S}(k)$ whose size is $|cl_{U}(k)| - |cl_{S}(k)|$.
Given the set of $k$ for which $h(k) = i$ the \emph{bad answers} are then
    \begin{equation}
    1 - \prob{\textrm{S hit the given cell}} = 1 - \left(1 - \frac{1}{m}\right)^{|S|}
    \end{equation}
We now provide a lower and upper bound for the said value.
The lower bound is given by the \emph{perfect hash} with no collisions: given
$e = $ number of $1$ in $B_s$, $e$ is the lowerbound.
Provided that  $m \geq c |S|$ for some $c > 1, |S| \leq \frac{m}{c}$:
    \begin{equation}
    \prob{\textrm{collision over i}} = \alpha_{S} = \frac{\frac{m}{c}}{m} = \frac{1}{c}
    \end{equation}
Therefore
    \begin{equation}
    e \leq 1 - \frac{1}{m^{|S|}} \leq \frac{1}{c}
    \end{equation}

\paragraph{Expected number of 1 in $B_{s}$.}
We define a random variable $X_{k}$:
    \begin{equation*}
    X_{k} = \begin{cases}
            1   & if B_{s}[k] = 1  \\
            0   & otherwise
            \end{cases}
    \end{equation*}
and one over the ones in $B_{s}$, $X$:
    \begin{equation}
        X = \sum_{k = 0}^{m - 1}X_k
    \end{equation}
We compute the relative expected value $\E{X}$:
    \begin{align*}
       \E{X} & = \E{\sum_{k = 0}^{m - 1}X_k} \\
       & = \sum_{k = 0}^{m - 1} \left( \prob{B_{s}[k] = 1} \right) \\
       & = \sum_{k = 0}^{m - 1} \left( \frac{\alpha}{m}\right) \leq \frac{m}{c}
   \end{align*}
where $\prob{B_{s}[k] = 1} = \frac{\alpha}{m}$ as $\alpha$ are the favourable cases
(i.e. the collisions for a generic bucket $B_s[i]$), and $m$ is the size of $B_s$.