\section{Count-min sketch: range queries}

Show and analyze the application of count-min sketch to range queries $(i,j)$ for computing  $\sum^j_{k=i} F[k]$. Hint: reduce the latter query to the estimate of just $t \leq 2\ log\ n$ counters $c_1,c_2,...,c_t$. Note that in order to obtain a probability at most $\delta$ of error (i.e. that $\sum^t_{l=1}c_l > \sum^j_{k=i}F[k] + 2\epsilon\ log\ n ||F||$), it does not suffices to say that it is at most $\delta$ the probability of error of each counter $c_l$: while each counter is still the actual wanted value plus the residual as before, it is better to consider the sum $V$ of these $t$ wanted values and the sum $X$ of these residuals, and apply Markov’s inequality to $V$ and $X$ rather than on the individual counters.

\vspace{1cm}
\noindent
\textbf{Solution.} A range $[a,b]$ is a dyadic range if it's length is a power of two ($l=2^y$), and begins at a multiple of its own length: $[j2^y+1, (j+1)2^y]$. For example, $[13,16]$ can be written as $[3\cdot 2^2+1,(3+1)\cdot 2^2]$. Any arbitrary range of size $s$ can be partitioned into $O(\log s)$ dyadic ranges, for example \cite{Cormode11}:
$$[18,38]=[18,18]\cup[19,20]\cup[21,24]\cup[25,32]\cup[33,36]\cup[37,38]$$

Let $n$ be the size of the implicit vector $F$ whose entries we want to approximate. The idea is to maintain a collection $C$ of $\log_2 n$ CM sketches, one for each set of dyadic ranges of length $2^y$ $\forall y\in [0, \log_2 n-1]$. The operations on $C$ becomes:
\begin{itemize}
  \item \textbf{Update} ($F[i] = F[i] + v$). Every sketch in $C$ is updated, since each point $1 \leq i \leq n$ is member of $\log_2 n$ dyadic ranges.
  \item \textbf{Range queries} ($\sum_{i=l}^rF[i]$). The range $[l,r]$ is partitioned into at most $2\log_2 n$ dyadic ranges. For each partition, a point query is made to the corresponding sketch in $C$; the (estimated) result of the range query is the sum of the point queries. See Figure \ref{figure:dyadic-ranges}.
\end{itemize}

The time to compute the estimate or to make an update is $O(\log n\log\frac{1}{\delta})$. The space used is $O(\frac{1}{\varepsilon}\log n\log\frac{1}{\delta})$, because each sketch requires $O(\frac{e}{\varepsilon}\ln\frac{1}{\delta})$ space \cite{Cormode05}.

Let $F[l..r]=\sum_{i=l}^rF[i]$ be the answer to the range query and $\tilde{F}[l..r]$ the estimate. The guarantees are:
\begin{itemize}
  \item $\tilde{F}[l..r] \geq F[l..r]$
  \item $\prob{\tilde{F}[l..r] > F[l..r]+2\varepsilon\log n\norm{F}} \leq \delta$
\end{itemize}

% TODO: proof

\begin{figure}[hbt]
  \centering
  \includegraphics[width=0.5\linewidth]{images/dyadic-ranges}
  \caption{A hierarchy of dyadic ranges. The leaves are the ranges of length $2^0$, while the root corresponds to the single range of length $2^3$. Each level of the tree can be seen as a CM sketch table. To estimate $\sum_{k=2}^8F[k]$, the range $[2,8]$ is decomposed into dyadic ranges $[2,2], [3,4], [5,8]$. Each node contains the sum of the values stored in its children. Red nodes are queried and their sum is returned. Adapted from \cite{Cormode11}.}
    \label{figure:dyadic-ranges}
\end{figure}
