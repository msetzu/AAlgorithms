\section{Special case of most frequent item in the stream}

Suppose to have a stream of $n$ items, so that one of them occurs $> \frac{n}{2}$
times in the stream.
Also, the main memory is limited to keeping just two items and their counters, plus
the knowledge of the value of $n$ beforehand.
Show how to find deterministically the most frequent item in this scenario.
\\ Hint: since the problem cannot be solved deterministically if the most
frequent item occurs $\leq \frac{n}{2}$ times, the fact that the frequency is
$> \frac{n}{2}$ should be exploited.

\vspace{0.5cm}
\paragraph{Solution.}
We'll use the notation of the previous section.
Before illustrating our solution we prove that the following hold:
    \begin{itemize}
    \label{frequence_dominance}\item \textbf{Given the $j^{1}$ element, $C(j^{1}) > C(j^{i}) \forall i \in S$.}
    \item Following the previous, $C(j^{i}) < C(j^{i}),  i > 1$ element.
    \item \textbf{There are at most $\frac{n}{2} - 1$ elements in S:} given that $j^{1}$
    appears at least $\frac{n}{2} + 1$ times, in the best case scenario every
    other $\frac{n}{2} - 1$ element is different, giving $\frac{n}{2} - 1$
    different elements.
    \label{sub-stream_dominance}~\item \textbf{$j^{1}$ has sub-stream dominance:}
    $\exists S' \subseteq S: j^{1} \textrm{ is the most frequent item} \in S'$:
    trivially, since it is the most frequent element, $S'$ is always a sub-stream
    dominant sub-stream.
    \end{itemize}

\paragraph{Recursive sub-streams} It is trivial to show that if we were to have
$\frac{n}{2}$ counters, $\Sigma^{\frac{n}{2}}_{i = 0, i \neq 1}(C(j^{i}) < C(j^{1})$
by~\ref{frequence_dominance}.
Though trivial, by combining it with~\ref{sub-stream_dominance} we can assert that
given $S' = S[0, \frac{n}{2}], S'' = S[\frac{n}{2}, n], k = \textrm{ number of elements } \in S$
    \begin{itemize}
    \item \textbf{$j^{1}$} is the sub-stream dominant element in either $S', S''$:
    by contradiction, let us assume that is false. Then
    $\exists j_{o} \neq j^{1}: C(j_{o})$ in $S' > C(j^{1}$, $\exists j_{o}: C(j_{o})$ in $S'' > C(j^{1}$;
    given that $S'S'' = S$, that would make $C(j_{o}) = C(j^{1})$: contradiction.
    \item More generally, given two sub-streams $S', S''$, the sub-stream
    dominant element of $S'S''$ is one of the sub-stream dominants in
    $S'$ or $S''$ and its frequencies are defined by
    $$e - k - 1\textrm{ in } S', \frac{n}{2} + 1 - (e - k - 1) \textrm{ in } S''$$
    where $e$ is an arbitrary number of appearances of $j^{2}: e < \frac{n}{2} + 1$
    The reader will see as the $+1$ in the right side guarantees the
    sub-stream dominant to be maximum.
    The sides can be swapped without hurting generality.
    \end{itemize}

In order to better illustrate, we provide an algorithm (the Boyer–Moore majority vote algorithm) that exploits the
sub-stream dominance.
Our idea is to \emph{go up} with our counter when we are fed an object we've
already met, and to \emph{go down} when we are fed an object different from
us.
Once we reach the bottom (0), we know by~\ref{sub-stream_dominance} that
we either reached zero for an object $\neq j^{1}$, and we can ignore it,
or we reached zero for $j^{1}$, that being the most frequent element will
have sub-stream dominance in the remaining sequence, which means it will
\emph{go upper} than any other element in that sub-stream.
Given that any other element has lower frequency, the most frequent of
them, $j^{2}$ will \emph{go down} $e < \frac{n}{2} + 1$ times.

\begin{algorithmic}[1]
  \Function{majority\_vote\_algorithm}{$S$}:
    \State $candidate \gets$ null
    \State $counter \gets 0$
    \ForAll{$object \in S$}
       \If{$counter = 0$}
         \State $candidate \gets object$ \Comment{$candidate$ is no more dominant, swap it}
       \ElsIf{$object = candidate$}
         \State $counter \gets counter + 1$
       \ElsIf{$object \neq candidate$}
         \State $counter \gets counter - 1$
       \EndIf
    \EndFor
    \State \Return{$candidate$}
  \EndFunction
\end{algorithmic}